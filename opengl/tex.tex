\documentclass[10pt]{article}
\usepackage[utf8]{inputenc}
\usepackage[T1]{fontenc}
\usepackage{amsmath}
\usepackage{amsfonts}
\usepackage{amssymb}
\usepackage[version=4]{mhchem}
\usepackage{stmaryrd}
\usepackage{bbold}

\title{1 Introduction }

\author{}
\date{}


\begin{document}
\maketitle
%-----------------------------------------------------------------------------------------------------
\section{Introduction}
In the early stages of the universe, galaxies were nonexistent. However, today, they exist in staggering numbers, reaching into the billions. The process of galaxy formation is understood by astronomers who apply the fundamental laws of physics to unveil its basic narrative. Galaxies originate from vast clouds of gas that undergo a process of collapse and rotation. During their evolution, stars take shape within these galaxies. The appearance of galaxies can be altered through collisions with one another. By examining the depths of space, astronomers observe galaxies in earlier developmental phases, providing insights into their evolution. These primordial galaxies are characterized by increased numbers, unique shapes, and periods of energetic outbursts. The observation of galaxies extends back more than 10 billion years, allowing scientists to trace their existence and transformations over cosmic timescales.

All the galaxy moving away from us implies Universe Expands (gravitational red shift ,blue shift , also galaxy ages , supernova 1a implies its picking up speed.Agent of acceleration has popped out very lately , Presence of Cosmic microwave Background(CMB)radiation and its uniformity reeks of another Dark Energy phase around the berth.The prediction of Quantum application also fits well with average properties what we see today .But stillthere are predictions to be verified and results which cust a shadow .
\section{Self Gravitational Bound:} 

Naturally question arises is astronomical objects bounded by gravitational pull and what holds things together? For the Sun its surface temperature is 5000K and the protons and plasma atoms has escape velocity around $10^{6} m/s$.In our solarsystem 
Moon is moving as 1 km/s planets average speed is 10km/s stars in Galaxy is 200-300 km/s .Galaxy clusters in a  supercluster also bound by self gravity, just like "Buckyballs" held by own gravitation pull ?

 Cepheid Variables can be under stood as  One Key to Cosmic Distances and pulsating variable stars, such as cepheids and RR Lyrae-type stars, where this class of variable stars, the time the star takes to go through a cycle of luminosity changes is related to the average luminosity of the star.Where RR Lyrae stars can be used as standard bulbs, and cepheid variables obey a period-luminosity relation,which used to calculate their distances(over 60 million light-years.) by comparing their luminosities with their apparent brightnesses,
 
 In the other hand the expansion and contraction of pulsating variables can be measured by using the Doppler effect. The lines in the spectrum shift toward the blue as the surface of the star moves toward us and then shift to the red as the surface shrinks back. As the star pulsates, it also changes its overall colour, indicating that its temperature is also varying as well as  the luminosity of the pulsating variable(most important for our purposes,) also changes in a regular way as it expands and contracts.

\section{Einstein's gravity :}
 
 If all clusters are independent we should expect further they are faster they moves which is unlikely to happen by chance but by stretching out spacetime fabric.
 If our Universe has nothing in it at all, no matter or energy of any form, you get the flat, unchanging, Newtonian space you’re intuitively used to: static, uncurved, and unchanging.If instead you put down a point mass in the Universe, you get space that’s curved: Schwarzschild space,And if you make it a little more complicated, by putting down a point mass that also rotates, you’ll get space that’s curved in a more complex way: according to the rules of the Kerr metric.The expansion of the universe is driven by all the mass, radiation and energy contained within it. The Friedmann equation, derived from Einstein’s famous equations for general relativity, can be used to predict how quickly the universe is expanding mathematically.


 Depending on the value of the Hubble constant(measure of expanding universe), this gives an age of about 14 billion years—not far off the current best-estimate of 13.8 billion years.The speeds of the farthest stars and galaxies that we can observe don’t match what the Hubble constant predicts. Because light from a distant object has traveled for billions of years to reach us, our observations are not only affected by the present-day value of the Hubble constant, but also what it was when the universe was expanding more slowly. In other words, the Hubble constant isn’t a constant at all!
Considering the metric :
- $ ds^ {2} $ =a $ (t)^ {2} $ $ ds_ {3}^ {2} $ - $ c^ {2} $ $ dt^ {2} $ 


$$
\frac{\dot{a}^2+k c^2}{a^2}=\frac{8 \pi G \rho+\Lambda c^2}{3},
$$
which is derived from the 00 component o
$$
\frac{\ddot{a}}{a}=-\frac{4 \pi G}{3}\left(\rho+\frac{3 p}{c^2}\right)+\frac{\Lambda c^2}{3}
$$

$$
R=\frac{6}{c^2 a^2}\left(\ddot{a} a+\dot{a}^2+k c^2\right)
$$

$$
\dot{\rho}=-3 H\left(\rho+\frac{p}{c^2}\right),
$$
which eliminates $\Lambda$ and expresses the conservation of mass-energy:
$$
T_{; \beta}^{\alpha \beta}=0 \text {. }
$$

These equations are sometimes simplified by replacing
$$
\begin{gathered}
\rho \rightarrow \rho-\frac{\Lambda c^2}{8 \pi G} \\
p \rightarrow p+\frac{\Lambda c^4}{8 \pi G}
\end{gathered}
$$
to give:
$$
\begin{aligned}
& H^2=\left(\frac{\dot{a}}{a}\right)^2=\frac{8 \pi G}{3} \rho-\frac{k c^2}{a^2} \\
& \dot{H}+H^2=\frac{\ddot{a}}{a}=-\frac{4 \pi G}{3}\left(\rho+\frac{3 p}{c^2}\right) .
\end{aligned}
$$
These are Friedmann Equations for a homogeneous and isotropic Universe .
Whatever goes around ... 
 • If the galaxy clusters are moving away and have been dong so in the past as well then they were 
 closer yesterday and day before —and century before 
 • One can compute what is the time it will take for a(t) to become O with the observed velocity which left us with an answer of 10B years
 • The universe collapses into a point 10 b yr ago (or it started from that point)
But Milky way itself is 13.6 B years old ,This much velocity should not have been there in past, if they move slowly in past , we get more time,which creates a problem.


SuperNova1a happens at known mass and known luminosity and thus gives result which looks like the expansion seep dips below Hubble,and cluster picked up  in recent time with accelerating pace.
There for the following gives detailed explanation of Modified Friedman Equation.
To make the solutions more explicit, we can derive the full relationships from the first Friedman equation:
$$
\frac{H^2}{H_0^2}=\Omega_{0, \mathrm{R}} a^{-4}+\Omega_{0, \mathrm{M}} a^{-3}+\Omega_{0, k} a^{-2}+\Omega_{0, \Lambda}
$$
with
$$
\begin{aligned}
H & =\frac{\dot{a}}{a} \\
H^2 & =H_0^2\left(\Omega_{0, \mathrm{R}} a^{-4}+\Omega_{0, \mathrm{M}} a^{-3}+\Omega_{0, k} a^{-2}+\Omega_{0, \mathrm{~A}}\right) \\
H & =H_0 \sqrt{\Omega_{0, \mathrm{R}} a^{-4}+\Omega_{0, \mathrm{M}} a^{-3}+\Omega_{0, k} a^{-2}+\Omega_{0, \Lambda}} \\
\frac{\dot{a}}{a} & =H_0 \sqrt{\Omega_{0, \mathrm{R}} a^{-4}+\Omega_{0, \mathrm{M}} a^{-3}+\Omega_{0, k} a^{-2}+\Omega_{0, \Lambda}} \\
\frac{\mathrm{d} a}{\mathrm{~d} t} & =H_0 \sqrt{\Omega_{0, \mathrm{R}} a^{-2}+\Omega_{0, \mathrm{M}} a^{-1}+\Omega_{0, k}+\Omega_{0, \Lambda} a^2} \\
\mathrm{~d} a & =\mathrm{d} t H_0 \sqrt{\Omega_{0, \mathrm{R}} a^{-2}+\Omega_{0, \mathrm{M}} a^{-1}+\Omega_{0, k}+\Omega_{0, \Lambda} a^2}
\end{aligned}
$$

Rearranging and changing to use variables $a^{\prime}$ and $t^{\prime}$ for the integration
$$
t H_0=\int_0^a \frac{\mathrm{d} a^{\prime}}{\sqrt{\Omega_{0, \mathrm{R}} a^{\prime-2}+\Omega_{0, \mathrm{M}} a^{\prime-1}+\Omega_{0, k}+\Omega_{0, \Lambda} a^{\prime 2}}}
$$

Solutions for the dependence of the scale factor with respect to time for universes dominated by each component can be found. In each we also have assumed that $Q_{0, k} \approx 0$, which is the same as assuming that the dominating source of energy density is approximately 1.

For matter-dominated universes, where $Q_{0, \mathrm{M}} \gg \Omega_{0, \mathrm{R}}$ and $\Omega_{0, A}$, as well as $\Omega_{0, \mathrm{M}} \approx 1$ :
$$
\begin{aligned}
t H_0 & =\int_0^a \frac{\mathrm{d} a^{\prime}}{\sqrt{\Omega_{0, \mathrm{M}} a^{\prime-1}}} \\
t H_0 \sqrt{\Omega_{0, \mathrm{M}}} & =\left.\left(\frac{2}{3} a^{\prime \frac{3}{2}}\right)\right|_0 ^a \\
\left(\frac{3}{2} t H_0 \sqrt{\Omega_{0, \mathrm{M}}}\right)^{\frac{2}{3}} & =a(t)
\end{aligned}
$$
which recovers the aforementioned $a \propto t^{\frac{2}{3}}$
For radiation-dominated universes, where $\Omega_{0, \mathrm{R}} \gg Q_{0, \mathrm{M}}$ and $\Omega_{0,1}$, as well as $Q_{0, \mathrm{R}} \approx 1$ :
$$
\begin{aligned}
t H_0 & =\int_0^a \frac{\mathrm{d} a^{\prime}}{\sqrt{\Omega_{0, \mathrm{R}} a^{\prime-2}}} \\
t H_0 \sqrt{\Omega_{0, \mathrm{R}}} & =\left.\frac{a^{\prime 2}}{2}\right|_0 ^a \\
\left(2 t H_0 \sqrt{\Omega_{0, \mathrm{R}}}\right)^{\frac{1}{2}} & =a(t)
\end{aligned}
$$
For $\Lambda$-dominated universes, where $\Omega_{0, \Lambda} \gg \Omega_{0, \mathrm{R}}$ and $\Omega_{0, \mathrm{M}}$, as well as $\Omega_{0, \Lambda} \approx 1$, and where we now will change our bounds of integration from $t_i$ to $t$ and likewise $a_i$ to $a$ :
$$
\begin{aligned}
\left(t-t_i\right) H_0 & =\int_{a_i}^a \frac{\mathrm{d} a^{\prime}}{\sqrt{\left(\Omega_{0, \Lambda} a^{\prime 2}\right)}} \\
\left(t-t_i\right) H_0 \sqrt{\Omega_{0, \Lambda}} & =\left.\ln \left|a^{\prime}\right|\right|_{a_i} ^a \\
a_i \exp \left(\left(t-t_i\right) H_0 \sqrt{\Omega_{0, \Lambda}}\right) & =a(t)
\end{aligned}
$$

The 1 -dominated universe solution is of particular interest because the second derivative with respect to time is positive, nonzero; in other words implying an accelerating expansion of the universe, making $\rho_{\Lambda}$ a candidate for dark energy:
$$
\begin{aligned}
a(t) & =a_i \exp \left(\left(t-t_i\right) H_0 \sqrt{\Omega_{0, \Lambda}}\right) \\
\frac{\mathrm{d}^2 a(t)}{\mathrm{d} t^2} & =a_i\left(H_0\right)^2 \Omega_{0, \Lambda} \exp \left(\left(t-t_i\right) H_0 \sqrt{\Omega_{0, \Lambda}}\right)
\end{aligned}
$$

Where by construction $a_i>0$, our assumptions were $\Omega_{0, \Lambda} \approx 1$, and $H_0$ has been measured to be positive, forcing the acceleration to be greater than zero.

\section{Acclerating Universe: }
If universe was born with a bang and constituents fields apart , the parts should stop after a while or keep moving at the speed they were born with.Gravity is repelling them apart,all other forces do not work over such length scales

• Accelerating universe : Requires exotic matter which exerts negative pressure : Dark energy 
 • Should account for the normal matter + exotic matter both in the evolution 
 • When we go back universe becomes smaller, density increases, wavelength gets shorter 
 • Thus energy density of normal matter shoots up, however for the exotic matter (dark energy) 
 the growth rate is marginal (does not change much) 
 • Finite life of the universe : Now corrected to 14 b yr
 • Thus towards the beginning energy density of usual matter was much large, DE was unable to 
 do anything 
 • But as the universe expanded the normal matter dilutes, DE remains practically unchanged and 
 is able to peep Its head up 
 • This is mysterious that it pops up just now
 • As photons become for energetic they rip 
 molecules and atoms apart 
 • The particles are to be •n bare 
 forms as we go back 
 Universe was a hot and dense object when it 
 was born 
 • Expansion cooled it down and atoms, 
 molecules, structure emerged.

\section{ Cosmic Microwave Background Radiation:}
 • As the universe expanded the photons lost energy 
 • After a while the photons are left with not sufficient energy to knock electrons out. 
 • They should free stream after that instant. Thus those photons from the last scattering should 
 be reaching us unhindered, carrying information upto the time they lost touch. This is called 
 Cosmic Microwave Background Radiation (CMBR). 
 • In they will affected by some gravity pull of galaxies and structures.. but that's about 
 it. No drastic effects !!
From the skymap by COBE,WMAP,PLANCK over years it became confirmed that • The CMB radiation is remarkably homogeneous and looks eerily similar to visible galaxy map around us .
 • However given finite age of universe one part of CME sky would not have got the chance to get 
 in touch with the other. 
 • Still all the parts know what temperature to maintain, how many galaxies to form - (who tells 
 where to a form a galaxy?)
 • It not only creates seeds for us, it creates all sort of particles - photons, gravitons 
 • Looks like the same computation which tells us about galaxy clusters falls short of generating 
 enough magnetic field ( problem of inflation or problem of electrodynamcs o.) 

 
 • Gravitational waves. Where are they ? 
 Do they also form black holes ? What happened to them ?
The horizon problem asks for an accelerating phase at the birth as well (another DE was hiding ) 
 The early DE phase is known as Inflation. 
 That largc acceleration phase needs to last for$ 10^{-35}5$ s. But it inflated the Size $ 10^{25}5$ fold. 
 Exciting proposal : This might have happened, but how do we know ? If universe was born this tiny, microscopic physics should have dictated.

\section{Quantum Theories and Probabilities:}
 • Quantum theory = Probabilistic outcomes .A probability theory would tell on the average what should you expect 
 For instance in coin toss it will not tell you what will be the exact outcome of the 4th or the 10th 
 toss  . Quantum laws tell you likelihood of happenings. 
 also For instance, what is the probability that there is no particle in thc universe 
 particle in the unverse (Extracting particles from quantum vacuum) 
In static case this probability goes to zero, however if things expands probability turns on ! If try to estimate average from probability , So rapid expansion should create many particles from the quantum fluctuations. 
 We do not know where exactly. It it would tell us average expectations. 
 These particles due to their mass and energy attract other particles. 
  For instance given a structure here , what is the chance of getting one there ? According to 
 quantum calculations that likelihood should have no scale dependency at large distance i.e. $k^{0}$ 
 • Observationally we find $k^{n+1}$  with n - 0.98 {error 0.031}
  Declaring victory with one computation may well be premature 
  So people come up with extracting more hidden properties. 
  But it is not tabletop experiment ..one needs sophisticated instruments with specific expertise 
 at right.

 \section{Conclusions:}
 Looking at the collection of galaxies around us, we have come to an understanding that the universe appears to be expanding. If we drag the clock backwards, it suggests that the universe should have been born a finite time ago as a very small and dense entity. We will discuss if any of the birthmarks of the Universe are still present for observations after having grown old for roughly 14 billion years. If they are, what is the story they tell us about its juvenile phase, and what puzzles do they leave us with?
 
\end{document}